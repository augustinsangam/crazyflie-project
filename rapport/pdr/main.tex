\documentclass{mistcoursedoc}
\usepackage[french]{babel}
\usepackage[utf8]{inputenc}
\usepackage{paralist}


\course{INF3995: Projet de conception d’un\\ système informatique}
\term{Hiver}
\termyears{2021} 
\doctype{Réponse à l’appel d’offres}

% Numéro équipe
\newcommand{\equipe}{203}

\author{Équipe \equipe}

\begin{document} 

\maketitle
\vspace{2cm}
\begin{center}

  {\Huge\bf Système aérien minimal pour exploration\\[3em]}

\Large Proposition répondant à l’appel d’offres  no. H2021-INF3995 du département GIGL\\[3em]


Équipe No. \equipe\\[3em]

  Bal, Samba Bousso\\[1em]
  Chritin, Mathurin\\[1em]
  Grootenboer, Hubert\\[1em]
  Maghni, Issam Eddine\\[1em]
  Sangam, Eya-Tom Augustin\\[1em]

\vfill

\end{center}

\section{Vue d’ensemble du projet}

\textit{Décrire brièvement le but et les objectifs du projet ainsi que des biens livrables attendus.}

\subsection{But du projet, porté et objectifs}

\par L'objectif principal de ce projet est de créer un système informatique de gestion de drones.
Le dit système permettra à son utilisateur d'explorer et cartographier un milieu arbitraire depuis une station au sol.
Le système pourra faire marcher ensemble une colonie de drones, communiquant entre eux, pour fournir une cartographie du milieu exploré.

\par Un tel système se vaudra très utile durant les explorations sur Mars, encore peu connue de tous.
Nous imaginons une situation dans laquelle, un robot plus complexe, 
plus lent et limité dans sa capacité de mouvement, se fera diriger par une colonie de drones qui
lui indiquera les endroits les plus intéressants à explorer.
\subsection{Hypothèse et contraintes}

\textit{Énumérer les hypothèses sur lesquelles repose ce plan ainsi que les contraintes dans le cadre de ce projet.  Pas seulement des éléments techniques, mais aussi des éléments externes à l’équipe.}

\subsection{Biens livrables du projet}

\textit{Énumérer les artéfacts qui devront être créés durant le projet avec leurs dates prévues de publication.}}
\begin{itemize}
  \item \textbf{Preliminary Design Review (PDR)} : prototype minimal afin de démontrer la maîtrise des technologies utilisées :
    \begin{itemize}
      \item Inter web minimale avec boutons « Take Off » et « Land », ainsi que le niveau de batterie, la position et la vitesse du drone en action dans le simulateur Argos 3
      \item Serveur maître faisant l'intermédiaire entre le simulateur et l'interface web. 
      \item Simulateur Argos faisant bouger deux drone.
    \end{itemize}
  \item \textbf{Critical Design Review (CDR)} : 
  \item \textbf{Readiness Review (RR)} : 
\end{itemize}

\section{Organisation du projet}

\subsection{Structure d’organisation}

\textit{Décrire la structure d’organisation de l’équipe de projet et les différents rôles des membres.}

\subsection{Entente contractuelle}

\textit{Décrire le type d’entente contractuelle proposée pour projet et les raisons de ce choix}

Un contrat de livraison clef en main serait adequat pour ce projet. En effet, le contracteur a une liste des requis complete et suffisament precise pour ne pas avoir a la modifier grandement au cours du projet.


\section{Solution proposée}

\subsection{Architecture logicielle générale}

\textit{Un diagramme qui résume l’architecture.  Un texte qui décrit et justifie les choix.

Inspiration: \og L’ingénieur a parfois un peu peur de réaliser des choses parce que les moyens sont maintenant considérablement sophistiqués. On oublie que seulement prendre un papier et un crayon, décrire les choses, faire une esquisse, cela peut être aussi valable qu’un dessin d’ordinateur.\fg{}}

On ajout l'image ici

\par La solution que nous avons retenue fais état de 3 entités :

\subsection{Architecture logicielle embarqué}

\textit{Un diagramme qui résume l’architecture.  Un texte qui décrit et justifie les choix.}


\subsection{Architecture logicielle station au sol}

\textit{Quelques blocs des principaux modules ou classes seulement.  Des diagrammes sont nécessaires.  Un texte qui décrit et justifie les choix.}


\section{Processus de gestion}

\subsection{Estimations des coûts du projet}

\textit{Les divers frais rattachés au projet peuvent nous aider à estimer le coût global du projet. La principale dépense n’est nul autre que les ressources humaines. Avec cinq ingénieurs à temps partiel, il est nécessaire d’avoir une estimation de temps pour en déduire le coût juste. Pour le bien de la cause, nous estimons 11 heures de travail par ingénieur par semaine. Sachant que nous disposons de cinq ingénieurs et que le projet s’échelonne sur 11 semaines, nous obtenons 605 heures pour la complétion du projet. En considérant que le salaire de nos ingénieurs avoisine les 100000\$ par an, le coût humain s’éleve à priori à 30250\$.}

\subsection{Planification des tâches}

\textit{Inclure: Un diagramme  indiquant l’allocation du temps pour chaque tâche. Fournir une vue d’ensemble de l’horaire des 2, 3, 4 (maximum) principaux jalons (milestones).  On doit aussi voir la répartition des tâches entre les membres de l’équipe.}

\subsection{Calendrier de projet}

\textit{Insérer un tableau qui indique les dates cibles de terminaison des phases importantes, des dates de version et autres jalons.  Un résumé seulement.}

\subsection{Ressources humaines du projet}

\textit{Indiquer le nombre et le type de ressources humaines nécessaires, incluant les qualifications spéciales ou l’expérience des membres de l’équipe.}

\section{Suivi de projet et contrôle}

\subsection{Contrôle de la qualité}

\textit{Tous les biens livrables doivent être soumis à un processus de révision. Une révision est requise afin de s’assurer, au moyen de lignes directrices et de listes de vérification, de la qualité de chaque bien livrable.}

\subsection{Gestion de risque}

\textit{Par exemple: Lister les principaux risques de ce projet et estimer leur importance.  Donner quelques solutions de remplacement possibles et la façon dont l’équipe entend gérer les changements en cours de projet.}

\subsection{Tests}

\textit{Identifier et préciser quelques tests pour chaque sous-système, tant pour le matériel que le logiciel.  Il devrait y avoir un lien entre ces tests et les tâches décrites plus haut.}

\subsection{Gestion de configuration}

\textit{Par exemple : Donner quelques renseignements sur le système de contrôle de version, l’organisation du code source, des tests et les fichiers de données ainsi que la documentation relative au code source et à la documentation de conception.  La séparation et l’intégration entre les fichiers de description du logiciel}

\textbf{Références} avec \texttt{printbibliography.}


\section*{ANNEXES}

\textit{Inclure toute documentation supplémentaire utilisable par le lecteur. Ajouter ou référencer toute norme technique de projet ou plans applicables au projet.}



\end{document}

%%% Local Variables:
%%% mode: latex
%%% TeX-master: t
%%% End:
